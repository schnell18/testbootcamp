%!TEX TS-program = xelatex
%!TEX encoding= UTF-8 Unicode

\documentclass[12pt,addpoints,fleqn]{exam}
\usepackage{ctex}
\usepackage{fontspec,xltxtra,xunicode}
\usepackage{listings,adjustbox}
\usepackage{graphicx,marvosym}
\usepackage[dvipsnames]{xcolor}
\usepackage{tikz}
\usepackage{minted}
\usepackage{booktabs,makecell,multirow,threeparttable,diagbox}
\graphicspath{{./images/}}
\XeTeXlinebreaklocale "zh"
\XeTeXlinebreakskip = 0pt plus 1pt minus 0.1pt
\usepackage{color}

\usepackage[margin=1in]{geometry}
\usepackage{amsmath,amssymb}
\usepackage{multicol}
\usepackage[
  colorlinks=true,
  linkcolor=blue,
  bookmarksnumbered=true,
  CJKbookmarks=true,
  bookmarksopen=true]{hyperref}

% setup water mark for solutions
\ifprintanswers
  \usepackage[draft,allpages]{draftmark}
  \draftmarksetup{showcenter=false,fontfamily=ppl,angle=10,scale=.8,
    color=red!10,coordunit=pc,xcoord=0,ycoord=0,
    mark={\fbox{*}\\[1ex]答案涉密\\请勿传播
    \\[1ex]\rotatebox{180}{\fbox{*}}}}
\fi

\newcommand{\class}{\raisebox{-0.7ex}{\includegraphics[scale=0.3]{logo}}技术委员会培训}
\newcommand{\term}{2023年度}
\newcommand{\examnum}{测试培训练习}
\newcommand{\examdate}{\today}
\newcommand{\timelimit}{60分钟}


% reduce choice indent
\renewcommand{\choiceshook}{%
    \setlength{\leftmargin}{2em}%
}

\newcommand\importJava[1]{\inputminted[breaklines,frame=leftline,framerule=1.1pt,framesep=1em,linenos,fontsize=\normalsize]{java}{#1}}
\newcommand\importJavaPlain[1]{\inputminted[breaklines,fontsize=\normalsize]{java}{#1}}
\newcommand\importSQLPlain[1]{\inputminted[breaklines,fontsize=\normalsize]{sql}{#1}}
\newcommand\importPython[1]{\inputminted[breaklines,frame=leftline,framerule=1.1pt,framesep=1em,linenos,fontsize=\normalsize]{python}{#1}}

\pagestyle{head}
\firstpageheader{}{}{}
\runningheader{\class\ - \examnum}{}{第\thepage 页,共\numpages 页}
\runningheadrule

\SolutionEmphasis{\itshape\small\color{red}}
\renewcommand{\questionlabel}{\textbf{第\hspace{-0.04ex}\thequestion\hspace{-0.02ex}题}}
\renewcommand{\solutiontitle}{\noindent\textbf{参考答案:}\par\noindent}

\begin{document}

\noindent
\begin{tabular*}{\textwidth}{l @{\extracolsep{\fill}} r @{\extracolsep{6pt}} l}
\textbf{\class} &\textbf{姓名:} & \makebox[2in]{\hrulefill}\\
\textbf{\term} &&\\
\textbf{\examnum} &&\\
\textbf{\examdate} &&\\
\textbf{考试时长: \timelimit} &\textbf{得分:}&\makebox[2in]{\hrulefill}
\end{tabular*}\\
\rule[2ex]{\textwidth}{2pt}

% TODO: use math expression to replace 80
% \newcounter{passingscore}
% \setcounter{passingscore}{80}
\noindent
本测验共有\numquestions 道题目,总分为\numpoints ,分为\numpages 页。
本测验必须\textbf{\emph{务必独立}}完成。

\begin{center}
  \textbf{评分表}\\
  \begin{tabular*}{0.75\textwidth}{@{\extracolsep{\fill}}|l*{10}{|c}|}
    \hline
    题目 & 01 & 02 & 03 & 04 & 05 & 06 & 07 & 08 & 09 & 10 \\
    \hline
    分数  & \pointsofquestion{1} & \pointsofquestion{2}
          &                      &                     
          &                      &                     
          &                      &                     
          &                      &                         \\
    \hline
    得分   &    &    &    &    &    &    &    &    &    &    \\
    \hline
    题目 & 11 & 12 & 13 & 14 & 15 & 16 & 17 & 18 & 19 & 20 \\
    \hline
    分数  &                       &                      
          &                       &                      
          &                       &                      
          &                       &                      
          &                       &                         \\
    \hline
    得分   &    &    &    &    &    &    &    &    &    &    \\
    \hline
    题目 & 21 & 22 & 23 & 24 & 25 & 26 &    &    &    &    \\
    \hline
    分数  &                       &                      
          &                       &                      
          &                       &                      
          &                       &
          &                       &                         \\
    \hline
    得分   &    &    &    &    &    &    &    &    &    &    \\
    \hline
  \end{tabular*}
\end{center}

\noindent
\rule[2ex]{\textwidth}{2pt}

\newpage

\begin{questions}
  \pointsinrightmargin
  \marginpointname{\hspace{-0.02ex}分}
  \setlength{\rightpointsmargin}{2cm}
  \CorrectChoiceEmphasis{\color{red}\itshape}

\newpage
\question[20] 斐波那契数列由公式\ref{eq:fibonacci}给出。
\begin{equation}
    \label{eq:fibonacci}
    f(n)=\left\{
        \begin{array}{l@{\quad:\quad}l}
            0  & n=0 \\
            1  & n=1 \\
            f(n-1) + f(n-2) & n>1
        \end{array}
    \right.
\end{equation}
请用尾递归(Tail Recursion)编写斐波那契数列计算程序,并用Benchmark模块对比普通递归和辗相加法的性能差异。
\fillwithdottedlines{\stretch{2}}
\ifprintanswers
\newpage
非递归Java程序实现如下:
\importJava{code/fibonacci.java}
验证的单元测试程序(JUnit 4)如下:
\importJava{code/fibonacci-test.java}
\fi

\newpage
\question[20] 
请用本次培训了解的知识和工具,设计一个适合你所负责产品的质量改进措施。
\fillwithdottedlines{\stretch{2}}

\end{questions}

\end{document}

% vim: set ai nu nobk expandtab sw=4 tw=72 ts=4 syntax=tex :
