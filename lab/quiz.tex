%!TEX TS-program = xelatex
%!TEX encoding= UTF-8 Unicode

\documentclass[12pt,addpoints,fleqn]{exam}
\usepackage{ctex}
\usepackage{fontspec,xltxtra,xunicode}
\usepackage{listings,adjustbox}
\usepackage{graphicx,marvosym}
\usepackage[dvipsnames]{xcolor}
\usepackage{tikz}
\usepackage{minted}
\usepackage{booktabs,makecell,multirow,threeparttable,diagbox}
\graphicspath{{./images/}}
\XeTeXlinebreaklocale "zh"
\XeTeXlinebreakskip = 0pt plus 1pt minus 0.1pt
\usepackage{color}

\usepackage[margin=1in]{geometry}
\usepackage{amsmath,amssymb}
\usepackage{multicol}
\usepackage[
  colorlinks=true,
  linkcolor=blue,
  bookmarksnumbered=true,
  CJKbookmarks=true,
  bookmarksopen=true]{hyperref}

% setup water mark for solutions
\ifprintanswers
  \usepackage[draft,allpages]{draftmark}
  \draftmarksetup{showcenter=false,fontfamily=ppl,angle=10,scale=.8,
    color=red!10,coordunit=pc,xcoord=0,ycoord=0,
    mark={\fbox{*}\\[1ex]答案涉密\\请勿传播
    \\[1ex]\rotatebox{180}{\fbox{*}}}}
\fi

\newcommand{\class}{\raisebox{-0.7ex}{\includegraphics[scale=0.3]{logo}}\,技术委员会培训}
\newcommand{\term}{2023年度}
\newcommand{\examnum}{UCloud测试实战练习}
\newcommand{\examdate}{\today}
\newcommand{\timelimit}{60分钟}


% reduce choice indent
\renewcommand{\choiceshook}{%
    \setlength{\leftmargin}{2em}%
}

\newcommand\importJava[1]{\inputminted[breaklines,frame=leftline,framerule=1.1pt,framesep=1em,linenos,fontsize=\normalsize]{java}{#1}}
\newcommand\importJavaPlain[1]{\inputminted[breaklines,fontsize=\normalsize]{java}{#1}}
\newcommand\importSQLPlain[1]{\inputminted[breaklines,fontsize=\normalsize]{sql}{#1}}
\newcommand\importPython[1]{\inputminted[breaklines,frame=leftline,framerule=1.1pt,framesep=1em,linenos,fontsize=\normalsize]{python}{#1}}

\newminted{golang}{
  autogobble,
  breaklines,
  frame=lines,
  framerule=0.8pt,
  framesep=1em,
  linenos,
  fontsize=\footnotesize
}

\pagestyle{head}
\firstpageheader{}{}{}
\runningheader{\class\ - \examnum}{}{第\thepage 页,共\numpages 页}
\runningheadrule

\SolutionEmphasis{\itshape\small\color{red}}
\renewcommand{\questionlabel}{\textbf{第\hspace{-0.04ex}\thequestion\hspace{-0.02ex}题}}
\renewcommand{\solutiontitle}{\noindent\textbf{参考答案:}\par\noindent}

\begin{document}

\noindent
\begin{tabular*}{\textwidth}{l @{\extracolsep{\fill}} r @{\extracolsep{6pt}} l}
\textbf{\class} &\textbf{姓名:} & \makebox[2in]{\hrulefill}\\
\textbf{\term} &&\\
\textbf{\examnum} &&\\
\textbf{\examdate} &&\\
\textbf{时长: \timelimit} &\textbf{得分:}&\makebox[2in]{\hrulefill}
\end{tabular*}\\
\rule[2ex]{\textwidth}{2pt}

% TODO: use math expression to replace 80
% \newcounter{passingscore}
% \setcounter{passingscore}{80}
\noindent
本练习共有\numquestions 道题目,总分为\numpoints ,分为\numpages 页。

\begin{center}
  \textbf{评分表}\\
  \begin{tabular*}{0.75\textwidth}{@{\extracolsep{\fill}}|l*{10}{|c}|}
    \hline
    题目 & 01 & 02 & 03 & 04 & 05 & 06 & 07 & 08 & 09 & 10 \\
    \hline
    分数  & \pointsofquestion{1} & \pointsofquestion{2}
          & \pointsofquestion{3} &                     
          &                      &                     
          &                      &                     
          &                      &                         \\
    \hline
    得分   &    &    &    &    &    &    &    &    &    &    \\
    \hline
    题目 & 11 & 12 & 13 & 14 & 15 & 16 & 17 & 18 & 19 & 20 \\
    \hline
    分数  &                       &                      
          &                       &                      
          &                       &                      
          &                       &                      
          &                       &                         \\
    \hline
    得分   &    &    &    &    &    &    &    &    &    &    \\
    \hline
    题目 & 21 & 22 & 23 & 24 & 25 & 26 &    &    &    &    \\
    \hline
    分数  &                       &                      
          &                       &                      
          &                       &                      
          &                       &
          &                       &                         \\
    \hline
    得分   &    &    &    &    &    &    &    &    &    &    \\
    \hline
  \end{tabular*}
\end{center}

\noindent
\rule[2ex]{\textwidth}{2pt}

\newpage

\begin{questions}
  \pointsinrightmargin
  \marginpointname{\hspace{-0.02ex}分}
  \setlength{\rightpointsmargin}{2cm}
  \CorrectChoiceEmphasis{\color{red}\itshape}

\newpage
\question[40] 斐波那契数列由公式\ref{eq:fibonacci}给出。
\begin{equation}
    \label{eq:fibonacci}
    f(n)=\left\{
        \begin{array}{l@{\quad:\quad}l}
            0  & n=0 \\
            1  & n=1 \\
            f(n-1) + f(n-2) & n>1
        \end{array}
    \right.
\end{equation}
在本练习中,你将尝试使用单元测试、基准测试和静态代码分析技术提升代码质量,
并通过代码测试覆盖率衡量代码被测试的程度。
我们将利用\emph{testbootcamp}项目进行上述练习。
该项目实现了计算斐波那契数列的多种算法以及部分单元测试,具体算法如下:
\begin{itemize}
    \item 辗转相加法(\emph{FibonacciIterative})
    \item 动态规划(\emph{FibonacciDynamicProgramming})
    \item 普通递归(\emph{FibonacciRecursive})
    \item 尾递归(\emph{FibonacciTailRecursive})
\end{itemize}

其中普通递归实现有缺陷,请运行单元测试并找出该缺陷,修正该程序,并确保所有单元测试都能通过。
尾递归(Tail Recursion)编写斐波那契数列计算程序缺少单元测试,请补充该算法的单元测试。
然后用Benchmark模块对比该算法和普通递归和辗相加法的性能差异。
请按以下步骤进行练习:
% \begin{golangcode}
% func Fibonacci3(n uint64) uint64 {
%     switch n {
%     case 0:
%         return 0
%     case 1:
%         return 1
%     default:
%         return fib(n, 2, 0, 1)
%     }
% }

% func fib(n, i, j, k uint64) uint64 {
%     if i == n {
%         return j + k
%     } else {
%         return fib(n, i + 1, k, j+k)
%     }
% }
% \end{golangcode}

\newpage
\begin{parts}
  \part{}准备工作

本练习基于github项目\href{https://github.com/schnell18/testbootcamp}{https://github.com/schnell18/testbootcamp}。\\
请确保你的电脑已经安装以下软件:

\begin{itemize}
    \item git
    \item golang (1.20+)
    \item 浏览器,推荐使用chrome
    \item 开发工具goland,或Vim及VS Code
\end{itemize}

运行以下命令克隆该项目:
\begin{itemize}
    \item \verb=cd ~/work=
    \item \verb=git clone https://github.com/schnell18/testbootcamp=
    \item \verb=cd ~/work/testbootcamp/golang=
\end{itemize}

MacOS用户运行\verb=brew install golangci-lint=命令安装golangci-lint。

  \part{}运行单元测试
  \begin{subparts}
    \subpart{}切换到testbootcamp项目的\verb=golang/fib=目录下
    \subpart{}执行\verb=go test -v .=
    \subpart{}观察输出
    \fillwithdottedlines{1in}
    \subpart{}是否观察到单元测试失败?为什么?
    \subpart{}修正缺陷并重新执行单元测试
  \end{subparts}

  \part{}运行Ginkgo单元测试
  \begin{subparts}
    \subpart{}切换到testbootcamp项目的\verb=golang/fib/bdd=目录下
    \subpart{}执行\verb=go test .=
    \subpart{}观察输出
    \fillwithdottedlines{1in}
  \end{subparts}

  \part{}运行所有单元测试
  \begin{subparts}
    \subpart{}切换到testbootcamp项目的\verb=golang=目录下
    \subpart{}执行\verb=go test -v ./...=
    \subpart{}观察输出
    \fillwithdottedlines{1in}
  \end{subparts}

  \part{}运行基准测试
  \begin{subparts}
    \subpart{}切换到testbootcamp项目的\verb=golang/fib/bench=目录下
    \subpart{}执行\verb#go test -bench=.#
    \subpart{}观察输出
    \fillwithdottedlines{1in}
    \subpart{}哪个实现的性能比较好?为什么?
    \fillwithdottedlines{1in}
  \end{subparts}

  \part{}测试覆盖率
  \begin{subparts}
    \subpart{}切换到testbootcamp项目的\verb=golang/fib=目录下
    \subpart{}执行\verb#go test –coverprofile=coverage.out#
    \subpart{}执行\verb#go tool cover -html coverage.out#
    \subpart{}观察在浏览器中的代码覆盖情况,并做简要记录
    \fillwithdottedlines{1in}
  \end{subparts}

  \part{}使用golangci-lint扫描代码
  \begin{subparts}
    \subpart{}切换到testbootcamp项目的\verb=golang=目录下
    \subpart{}执行\verb#golangci-lint run#
    \subpart{}观察执行结果 \fillwithdottedlines{1in}
    \subpart{}请更正该问题后再次扫描
  \end{subparts}


  \part{}补充单元测试及基准测试
  \begin{subparts}
    \subpart{}补充尾递归算法的单元测试及基准测试
    \subpart{}重复之前的步骤运行单元测试和基准测试
    \subpart{}观察输出
    \fillwithdottedlines{1in}
    \subpart{}尾递归算法的性能是否有提升?为什么?
    \fillwithdottedlines{1in}
  \end{subparts}

  \part{}基准测试可以帮忙我们了解程序耗时的分布情况。本部分收集动态规划算法的CPU profile
  \begin{subparts}
    \subpart{}运行命令: \\
      \verb|go test -run=NONE \| \\
      \verb|        -bench='BenchmarkFibonacciDynamicProgramming$' \| \\
      \verb|        -cpuprofile=dp.out| \\
      \verb|        -count=5|
    \subpart{}然后运行\verb|go tool pprof -pdf dp.out|
    \subpart{}打开上一步生成的PDF文件,并观察记录:
    \fillwithdottedlines{1in}
  \end{subparts}

  \part{}模糊测试可以帮忙我们进行高强度的测试覆盖。\emph{testbootcamp}项目中的\emph{Reverse}
  函数的单元测试虽然可以通过,但该函数未能正确处理多字节字符串,请运行模糊测试并找出bug。
  \begin{subparts}
    \subpart{}运行命令: \\
      \verb|cd golang/reverse  \| \\
      \verb|go test -fuzz=Fuzz| \\
    \subpart{}观察输出是否有\emph{Failing input written to
      testdata/fuzz/FuzzReverse}字样?
    \fillwithdottedlines{1cm}
    \subpart{}观察当前目录下是否生成了testdata目录,运行命令: \verb|find testdata| 加以验证,简述一下你的发现:
    \fillwithdottedlines{1in}
    \subpart{}运行命令: \verb|go test| 观察输出,简述一下你的发现:
    \fillwithdottedlines{1in}
  \end{subparts}

\end{parts}

\newpage
\question[40] 
自定义数据(UserData)是指主机初次启动或每次启动时,系统自动运行的配置脚本。
用户可通过控制台或 API 传入该数据,并由主机内的 cloud-init 程序执行。
只有镜像中安装了 cloud-init 程序的主机才支持用户自定义数据,
控制台主机创建页面才会展示“自定义数据”选项。
请针对以下功能点,编写对应的测试用例:

\begin{itemize}
  \item 脚本内容不能超过 16 KB;
  \item 初次启动和每次开机 / 重装 / 重启时执行;
  \item 自定义数据支持编辑修改;
  \item 可在主机内部获取用户自定义数据 `curl http://100.80.80.80/user-data`
\end{itemize}
本次测试任务只需考虑已经支持cloud-init的官方CentOS最新版本镜像,
请按照本题所列步骤完成测试用例、测试评审、测试执行和测试报告。

\begin{parts}
  \part{}准备工作

  请用邮箱在\href{https://www.overleaf.com}{https://www.overleaf.com} 注册一个overleaf帐号。
  
  \part{}登录Metersphere测试平台

  请用公司帐号登录Metersphere测试平台。\\
  该系统的地址是:\href{https://ut.ucloudadmin.com/metersphere}{https://ut.ucloudadmin.com/metersphere}
  该系统集成了公司的统一登录系统CAS,你可以免密登录。
  
  \part{}进入新手训练营项目
  \begin{subparts}
    \subpart{}使用右上角工作空间下拉框切换到\emph{bootcamp}工作空间
    \subpart{}点击顶部\emph{测试跟踪}标签页,使用左上角项目下拉框切换到\emph{FirstClass}项目
  \end{subparts}

  \part{}创建专属测试模块
  \begin{subparts}
    \subpart{}点击顶部第二排\emph{功能用例}标签页
    \subpart{}将鼠标停留在左侧导航栏的\emph{全部用例}节点片刻,并在出现加号后点击加号
    \subpart{}输入\emph{新手训练-本人姓名},添加一个专属测试模块
  \end{subparts}

  \part{}编写测试用例
  \begin{subparts}
    \subpart{}选中刚刚创建的模块,点击\emph{更多操作}下拉菜单后再点击点击\emph{新建用例}菜单
    \subpart{}在\emph{新建用例}对话框中输入用例名称后点击\emph{编辑详情}按钮进入用例编辑界面
    \subpart{}根据测试要求在\emph{步骤信息}中填写详尽的\emph{前置条件}和\emph{步骤描述}
  \end{subparts}

  \part{}创建测试用例评审
  \begin{subparts}
    \subpart{}点击顶部第二排\emph{用例评审}标签页
    \subpart{}点击\emph{创建用例评审}按钮,进入\emph{创建用例评审}对话框
    \subpart{}填写\emph{评审名称}、\emph{评审人}和\emph{截止时间}后点击\emph{规划\&执行}按钮
    \subpart{}在用例评审界面中点击\emph{关联用例}按钮,在\emph{关联用例}对话框选择相关用例
  \end{subparts}

  \part{}评审用例
  \begin{subparts}
    \subpart{}点击顶部第二排\emph{用例评审}标签页
    \subpart{}在\emph{评审}下拉框中选择上一步创建的用例评审
    \subpart{}点击\emph{开始评审}按钮,进入评审界面
    \subpart{}使用右侧评论区记录评审过程中的建议,并评审意见标记该用例是否评审通过
    \subpart{}修改未评审通过的用例后再次评审直至通过
  \end{subparts}

  \part{}创建测试计划
  \begin{subparts}
    \subpart{}点击顶部第二排\emph{测试计划}标签页
    \subpart{}点击\emph{创建测试计划}按钮,进入\emph{创建测试计划}对话框
    \subpart{}填写\emph{计划名称}、\emph{负责人}、\emph{测试阶段}、\emph{计划开始}、\emph{计划结束}后点击\emph{规划\&执行}按钮
    \subpart{}在测试计划界面中点击第三排\emph{功能用例}标签页,点击\emph{关联测试用例}按钮
    \subpart{}在\emph{关联测试用例}对话框选择相关用例
  \end{subparts}

  \part{}执行测试
  \begin{subparts}
    \subpart{}点击顶部第二排\emph{测试计划}标签页
    \subpart{}在\emph{计划}下拉框中选择上一步创建的测试计划
    \subpart{}点击第三排\emph{功能测试用例}标签页,选择列表中的测试用例,进入测试执行详情页
    \subpart{}检查前置条件中的要求是否已满足,否则进行相应的设置或准备工作
    \subpart{}执行每个测试步骤,并记录实际结果和是否该步骤通过测试
    \subpart{}如有缺陷,则提交缺陷并和测试用例关联
    \subpart{}标记测试用例是否通过
    \subpart{}完成该测试计划中的所有测试
  \end{subparts}

  \part{}生成测试报告
  \begin{subparts}
    \subpart{}点击顶部第二排\emph{测试计划}标签页
    \subpart{}在测试计划列表中找到上个步骤中操作过的测试计划,在操作栏点击发送报告(纸飞机图标)
    \subpart{}编辑\emph{报告总结},添加必要的描述后进行保存
    \subpart{}点击右上角\emph{发送}按钮,在\emph{发送报告配置}对话框中按需设置抄送人员等设置后点击\emph{发送}按钮
    \subpart{}查看邮件
  \end{subparts}

  \part{}生成测试报告PDF版本
  \begin{subparts}
    \subpart{}点击顶部第二排\emph{报告}标签页
    \subpart{}在测试计划列表中找到上个步骤中操作过的测试计划,在操作栏点击查看测试报告(文档图标)
    \subpart{}在\emph{原始报告}页的右上角点击\emph{LaTeX}下载按钮
    \subpart{}等待浏览器下载完测试报告LaTeX源代码的zip文件
    \subpart{}在overleaf中创建一个新项目,并上传上个步骤中下载的zip文件
    \subpart{}上传完成后,overleaf会自动编译测试报告,如果右侧预览的PDF未能正确生成,请点击最右上角的\emph{Menu},并在弹层的\emph{Settings}区域下的\emph{Compiler}下拉框中选择\emph{XeLaTeX}选项后重新编译
    \subpart{}浏览测试报告内容,按需调整或添加你需要的内容并重新编译
    \subpart{}下载PDF文件,将测试报告存档
  \end{subparts}

\end{parts}
\newpage
\question[20] 
请用本次培训了解的知识和工具,设计一个适合你所负责产品的质量改进措施。
\fillwithdottedlines{\stretch{1}}

\end{questions}

\end{document}

% vim: set ai nu nobk expandtab sw=4 tw=72 ts=4 syntax=tex :
